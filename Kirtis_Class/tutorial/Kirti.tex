\providecommand{\exclude}[1]{}
% Simple approach for selecting styles
% #1 article + archive
% #2 revtex
% #3 Tufte
\newcommand{\select}[3]{#3}
\select
 {% #1 article + preprint
}{% #2 revtex
}{% #3 Tufte
} 

\select
 {% #1 article + archive
  \documentclass[%twocolumn,
                 switch]{article}
  \usepackage{arxiv}
  \usepackage{caption}
  \usepackage{authblk}
  \bibliographystyle{plain}
}{% #2 revtex
  \documentclass[10pt, prl, superscriptaddress, nofootinbib,
                 amsmath,   % amsfonts, amsmath, amssymb,
                 twocolumn, % preprintnumbers,
                 raggedbottom,
                 floatfix,
                 longbibliography, %noeprint,
                  colorlinks=true, linkcolor=blue, citecolor=blue, urlcolor=blue,
                 ]{revtex4-2}
}{% #3 Tufte
  \documentclass[notitlepage, justified, openany]{tufte-book} % notitlepage
  \usepackage[caption=false]{subfig}
  \bibliographystyle{unsrt} 
  \usepackage{fancyhdr} 
  \pagestyle{fancy}
  \renewcommand{\sectionmark}[1]{\markright{#1}} % remove section number from header of pages.
\renewcommand{\chaptermark}[1]{\markboth{#1}{#1}}
\fancyhead[RO]{\rightmark\  \ \  \ \  \thepage} 
\fancyhead[LE]{ \thepage\   \ \  \ \  \leftmark }
} 

\providecommand{\captionsetup}[1]{}

\usepackage[nottoc]{tocbibind} % Remove Chapter number from References
\usepackage[colorinlistoftodos, prependcaption, textsize=small, color=Dandelion]{todonotes}


%______________________________________________________________________
%%
%%       preamble 
%%
\usepackage{etoolbox}
\usepackage{amssymb}
\usepackage{epsfig}
\usepackage{amsmath}
\usepackage{amsthm}
\usepackage{url}
\usepackage{color}
\usepackage{graphicx}
\graphicspath{{figures/}}
\usepackage{makeidx}
\usepackage{relsize}
% \usepackage{titlesec}
\usepackage{tcolorbox}
\usepackage{array}
\usepackage{tabularx}
\usepackage{colortbl}
\usepackage{xcolor}
\usepackage{multirow} % needed for cline for example

%% HN packages
% \usepackage{amsfonts}
% \usepackage{tocloft} % trying to put dots in table of content. goes with \renewcommand{\cftdot}{} before \tableofcontents. In Tufte case removing the \renewcommand{\cftdot}{} will generate the dots.
\usepackage{enumitem,mathtools}
% \usepackage[ruled,vlined]{algorithm2e}
\usepackage[ruled,vlined,resetcount,algochapter]{algorithm2e}
\usepackage{algorithmic}
%\usepackage{algorithm}
%\usepackage{algpseudocode}

% \usepackage{soul}
% \modulolinenumbers[5]

% for flowchart
\usepackage{tikz}
\usetikzlibrary{shapes.geometric, arrows}
\usepackage{soul} % for highlightinh

% break the damn long references (URL link)
\usepackage{lineno}
\usepackage{url}
\def\UrlBreaks{\do\/\do-}
\usepackage{breakurl}
% \usepackage[breaklinks]{hyperref}

% \usepackage{comment}
% \renewenvironment{comment}{}{}

\providecommand{\bmmax}{0}
\RequirePackage{bm}
  
% \usepackage[margin=1.5in]{geometry}
%\setlength{\textwidth}{6.0in}
%\setlength{\textheight}{7.0in}
%\setlength{\topmargin}{0.5in}
%\setlength{\oddsidemargin}{0.0in}
%\setlength{\evensidemargin}{0.0in}


\DeclareMathOperator{\transverse}{\cap\kern-7.75pt\top}

\definecolor{darkpastelpurple}{rgb}{0.59, 0.44, 0.84}
\def\code#1{\textbf{\texttt{#1}}}
\def\codeRed#1{{\color{Maroon}{\textbf{\texttt{#1}}}}}
\def\codeGreen#1{{\color{Green}{\textbf{\texttt{#1}}}}}
\def\codeCyan#1{{\color{cyan}{\textbf{\texttt{#1}}}}}
\def\vari#1{{\color{Cerulean}{\textbf{\texttt{#1}}}}}
\def\func#1{{\color{Purple}{\textbf{\texttt{#1}}}}}
\newenvironment{coded}{\color{blue}\code}

\definecolor{titlered}{rgb}{.6,0,0}
\definecolor{titleblue}{rgb}{0,0,0.6}
\definecolor{exerciseblue}{rgb}{0,0,.6}
\definecolor{proofgreen}{rgb}{0.6,0,0}
\definecolor{myblue}{rgb}{0,0,0.7}
\definecolor{mymagenta}{rgb}{0.7,0,0}
\definecolor{aliceblue}{rgb}{0.94, 0.97, 1.0}
\colorlet{shadecolor}{gray!40}
\definecolor{color2}{RGB}{1,2,3}

\newcommand{\EVI}{E\!V\!I}
\newcommand{\NDVI}{N\!D\!V\!I}
\newcommand{\NIR}{N\!I\!R}

\newcommand{\df}{d\!f}
\newcommand{\tblue}{\color{titleblue}}
\newcommand{\tred}{\color{titlered}}
\newcommand{\pgreen}{\color{proofgreen}}
\newcommand{\green}{\color{green}}
\newcommand{\blue}{\color{blue}}
\newcommand{\red}{\color{red}}
\newcommand{\magenta}{\color{magenta}}

\newcommand{\Rank}{\operatorname{rank}}
\newcommand{\spn}{\operatorname{span}}
\newcommand{\Cyl}{\operatorname{Cyl}}
\newcommand{\T}{\operatorname{T}}
\newcommand{\D}{\operatorname{D}}
%\newcommand{\argmin}{\operatorname*{argmin}}
%\newcommand{\argmax}{\operatorname*{argmax}}
\DeclareMathOperator*{\argmax}{argmax}
\DeclareMathOperator*{\argmin}{argmin}
\DeclareMathOperator{\match}{match}
\newcommand{\clos}{\operatorname{clos}}
\newcommand{\bdy}{\operatorname{bdy}}
\newcommand{\diam}{\operatorname{diam}}
\newcommand{\Lip}{\operatorname{Lip}}
\newcommand{\ga}{\operatorname{ga}}
\newcommand{\dis}{operatorname{dis}}
\newcommand{\dist}{\operatorname{dist}}
\newcommand{\vol}{\operatorname{vol}}
\newcommand{\epi}{\operatorname{epi}}
\newcommand{\epo}{\operatorname{epo}}
\newcommand{\cnv}{\operatorname{cnv}}
\newcommand{\R}{\Bbb{R}}
\newcommand{\Z}{\Bbb{Z}}
\newcommand{\bbeta}{\bm \beta}
\newcommand{\hatM}{{(\bm X^T \bm X)^{-1} \bm X^T}}
%\newcommand{\R}{\mathbf{R}}
%\newcommand{\R}[1]{\mathbb{R}^{#1}}
\newcommand{\Hd}{\mathcal{H}}
\newcommand{\F}{\Bbb{F}}
\newcommand{\CHi}[1]{\LARGE \chi_{\tiny #1}}
%\newcommand{\CHI}{{\LARGE \chi}}
\newcommand{\eps}{\varepsilon}
\newcommand{\del}{\delta}
\newcommand{\1}{\mathbf{1}}
\newcommand{\Div}{\mbox{div}}
\newcommand{\done}{$\Box$}
\newcommand{\Per}{\mbox{Per}}
\newcommand{\Ar}{\mbox{Area}}
\newcommand{\myb}[1]{{\bf #1}}
\newcommand{\op}[2]{{\bf #1}\,#2}
\newcommand{\X}[1]{\mathcal{X}(#1)}
\newcommand{\lip}{\mbox{Lip}}
\newcommand{\Rto}[2]{\R^{#1}\rightarrow\R^{#2}}
\newcommand{\ipd}[2]{\left<#1,#2\right>}
\newcommand{\nind}{\noindent}
\newcommand{\Ln}[1]{\mathcal{L}^{#1}}
\newcommand{\Le}{\mathcal{L}}
\newcommand{\mychi}[1]{\mathlarger{\mathlarger{\mathlarger{\chi}}}_{\mathsmallerg{\mathsmaller{#1}}}}
\newcommand{\mychitest}[3]{\text{\scalebox{#1}{$\chi$}}_{\text{\scalebox{#2}{$#3$}}}}
\newcommand{\myell}{\,\hbox{\vrule height 6pt depth 0pt
\vrule height 0.4pt depth 0pt width 5pt}\,}
\newcommand{\mylee}{\,\hbox{\vrule height 0.4pt depth 0pt width 5pt
\vrule height 6pt depth 0pt}\,}
%\newcommand{\myfig65}[1]{\includegraphics[width=0.65\textwidth]{#1}}

\newcommand{\warn}{\textbf{\color{red}{Warning }}}

\newcommand{\abs}[1]{\lvert{#1}\rvert}
\newcommand{\norm}[1]{\lVert{#1}\rVert}
\newcommand{\mat}[1]{\bm{#1}}
\newcommand{\vect}[1]{\vec{#1}}
\newcommand{\braket}[1]{\langle{#1}\rangle}

\newtheorem{thm}{Theorem}[section] % "[section]" restarts the theorem counter at every new section.
\newtheorem{Coro}[thm]{Corollary}
\newtheorem{lem}[thm]{Lemma}
\newtheorem{rem}[thm]{Remark}
\newtheorem{exm}[thm]{Example}
\newtheorem{deff}[thm]{Definition}
\newtheorem{exc}[thm]{Exercise} % "[thm]" in the middle will use the same counter as the theorem environment. if "[thm]" was at the end, then  the counter of this new environment will be reset every time a new theorem environment is used.
\newtheorem{claim}[thm]{Claim}
\newtheorem{cond}[thm]{Condition}

\newcommand{\Tau}{\mathcal{T}}
\renewcommand\qedsymbol{$\blacksquare$}



%revtex uses \prb...
%\newtheorem{prb}{Problem}[section]
\newtheorem{excse}{Exercise}[section]
\newcommand{\exercise}[2]{\begin{excse}\label{#1}\color{exerciseblue}\emph{#2}\end{excse}}
\newcommand{\idef}[1]{\index{#1}{\bf\emph{#1}}}

\newcommand{\intrr}{\operatorname{int}}

\newcommand{\ifft}{if and only if }

% flowchart definitions
\tikzstyle{startstop} = [rectangle, rounded corners, minimum width=3cm, minimum height=1cm,text centered, draw=black, fill=red!20]

\tikzstyle{io} = [trapezium, trapezium left angle=0, trapezium right angle=0, minimum width=1cm, minimum height=1cm, text centered, draw=black, fill=blue!10]

\tikzstyle{process} = [rectangle, minimum width=3cm, minimum height=1cm, text centered, draw=black, fill=orange!30]
\tikzstyle{decision} = [diamond, minimum width=3cm, minimum height=1cm, text centered, draw=black, fill=red!70]

\tikzstyle{arrow} = [thick,->,>=stealth]

%\usepackage{mathptmx}
%\renewcommand{\familydefault}{\sfdefault}
%\renewcommand{\familydefault}{\ttdefault}
\usepackage{hyperref}
\hypersetup{
    colorlinks=true,
    linkcolor=cyan, % color1 : will be black
    filecolor=cyan,
    urlcolor=proofgreen,
    citecolor=red,
    %bookmarks=true, % true means bookmarks
     bookmarksnumbered=false, % left window are numbered
    bookmarksopen=false,
    pdftitle={Title},
    pdfauthor={Author},
}

\usepackage[nameinlink]{cleveref}
  % \cref{} and \Cref{} which work like \autoref but also
  % allow for capialization.  The 'nameinlink' used to mimic \autoref's behavior
  % of including the cross-reference's name in the link
  % https://tex.stackexchange.com/a/153480/6903

  % These should be capitalized everywhere.
  \crefname{table}{Table}{Tables}
  \crefname{figure}{Fig.}{Figs.}
  \crefname{equation}{Eq.}{Eqs.}


\newenvironment{mylist}{\begin{list}{*}{\itemsep=.01in \parsep=.01in
                  \topsep=.1in \parskip=.01in}}{\end{list}}
\newenvironment{mybiglist}{\begin{list}{*}{\itemsep=.1in \parsep=.1in
                  \topsep=.2in \parskip=.1in}}{\end{list}}
\newenvironment{myproof}[1][\proofname]{\proof[#1]\mbox{}\\*}{\endproof}

\setlist[description]{%
  topsep=10pt,               % space before start / after end of list
% itemsep=5pt,               % space between items
%   font={\bfseries\sffamily}, % set the label font
%  font={\bfseries\sffamily\color{red}}, % if colour is needed
font={\color{blue}},
}

\select{% arXiv argumenths, nothing here
}
{% revtex argumenths, nothing here
}
{ % Tufte argumenths
% \titleformat{\chapter}
% {\color{red}\normalfont\LARGE\bfseries}
% {\color{red}\thechapter}{2em}{}

\titleformat{\chapter}[display]
{\color{titlered}\normalfont\large\bfseries}
{\color{titlered}\chaptertitlename\ \thechapter}{20pt}{\Huge}

\titleformat{\section}
{\color{titleblue}\normalfont\LARGE\bfseries}
{\color{titleblue}\thesection}{1em}{}
}%select

%%%%%%%%%%%%%%%%%%%%%%%%%%%%%%%%%%%%%%%%%%%%%%%%%%%%%%%%%
%\LogoOff  %this is for the foiltex documents

\begin{document}
%\raggedright

\title{Kirti's Class}
\date{\today}

\select
{% #1 article + archive
  \renewcommand*{\Authfont}{\bfseries}
  \author[1]{Hossein Noorazar}
    \affil[1]{Center for Geometric Analysis and Data and
               Department of Mathematics, Washington State University}
}{% #2 revtex
  \author{Hossein Noorazar}
  \affiliation{Center for Geometric Analysis and Data and
                  Department of Mathematics Washington State University}
}{% #3 Tufte
  \author{Hossein Noorazar}
  \patchcmd{\maketitle}{\newpage}{}{}{}
  \let\oldmaketitle\maketitle
  \def\maketitle{\begin{fullwidth}\oldmaketitle\end{fullwidth}}
} 

\providecommand{\marginnote}[1]{Sidebar: {#1}}

\maketitle
\setcounter{secnumdepth}{3} % If this commented out. Chapters/Sections are not numbered
\setcounter{tocdepth}{3} % If this commented out. Sections are not included in ToC
\newpage

\tableofcontents


\newpage
\chapter{Coding}
\label{chap:Coding}


\begin{rem}
Use meaningful names. Refrain from 
using \code{v}, then double v; \code{vv},
and then \code{w}.
\end{rem}


\begin{rem}
Refrain from copying your code from
one file to another.
\end{rem}
\noindent When you copy your code from one file/module
to another, it means you use it a lot. Thus, you can
save it in a unique module and call it again and again.
This is easier, faster, and more importantly sustainable.
For example, if you plot a vector with your favorite settings,
write a function once then use it. If later you need
to change the font to be consistent with
the journal you want to submit your paper to, then, you can
change the font only once. Otherwise, you have to go to
all the files you wrote before and change them one at a time!
Just stop copy/pasting.

In~\cref{fig:FunctionDocumentation} we see how a function is defined
and documented in Python.
A few points to make here using this example are given below.
\begin{figure}
  \centering
  \includegraphics[width=1\textwidth]{figures/function_documentation}
  \caption{Function Documentation.}
  \label{fig:FunctionDocumentation}
\end{figure}

\noindent Remarks about the function in~\cref{fig:FunctionDocumentation};
\begin{description}
\item [Function Name] is meaningful.
\item [Input Variables] have meaningful name. 
\item [First Input Variable] There are different conventions. In this
example we see the first input variable is named \code{prices\_arr}.
We see that not only the variable name is meaningful but also
it is implying the type of the variable. It is an array. It is not a list.
It is not a panda dataframe. 

\item [Variables with Default Values] come last.

\item [Function's Description] Include the functions' descriptions
the first thing in the function. The first line (text) can describe what the function does. 
Then, define the arguments; input variables. 
We have \code{prices\_arr : array like} and then in the line below \code{Stock prices: `prices\_arr.shape = (Ndays, Nstocks)'}.
So, immediately after the variable name its type is specified. In the line that follows we
say what it is representing.

\item [Include Example(s)] You can include an example at the end of the
description to show how the function is used. This is not necessary unless
later you want to share your code, form a library and release it.
Then, you can have pages
like Python's documentation pages.

\item [Make Comments] throughout the code and functions.
Write down what each line is doing or why. You can skip
simple stuff. If you do not make comments and need to use
the code after a month you may have hard time remembering.
\end{description}
The example above is on \href{https://github.com/HNoorazar/Ag/blob/master/Kirtis\_Class/Python/Kirtis\_Class\_core.py}{GitHub} for you to review.
The convention for documentation used in this example is the same as 
that of NumPy~\citep{numpyStyle}.

Once you write your functions in a central module
you can import them into your driver modules and use
them the way you import standard libraries/packages of Python or R.

Say we have two modules that are the engine of our project.
\marginnote{I use sublime text editor to write my codes}
One includes all the functions that generate stuff and the other
includes all the plotting functions; \code{kirtis\_class\_core.py},
\code{kirtis\_class\_plot\_core.py}.
Say these two modules are saved in 
the following directory:
\code{/Users/Documents/project1/}.
Then we can import these modules into our driver
and use them like so:

\begin{tcolorbox}
  \begin{algorithm}[H]
  \label{alg:importPythonCode}
   \caption{Import Your Python Modules.}
\SetAlgoLined
0. \codeGreen{import} \code{sys}\;
1. \code{sys.path.append(}'\codeRed{/Users/Documents/project1/}'\code{)}\;
2. \codeGreen{import} \code{kirtis\_class\_core} \codeGreen{as} \code{kcc} \;
3. \codeGreen{import} \code{kirtis\_class\_plot\_core} \codeGreen{as} \code{kcpc} \;
4. Use the function \codeCyan{diversify} from \code{kirtis\_class\_core.py} like \\
\code{diversified\_portfolio = kcc.\codeCyan{diversify}(\codeGoldenrod{x}, \codeGoldenrod{y}, \codeGoldenrod{z})}
\end{algorithm}
\end{tcolorbox}
A simple code demonstrating this is on GitHub that you can look at;
\href{https://github.com/HNoorazar/Ag/tree/master/Kirtis\_Class/Python}{github.com/HNoorazar/Ag/tree/master/Kirtis\_Class/Python}.\\

In a similar way you can import your R modules into the drivers.
Say your R module with all the functions in it is called \code{kirtis\_class\_core.R}
located in \code{/Users/Documents/project1/}.
You can import it like so in your driver:

\begin{tcolorbox}
  \begin{algorithm}[H]
  \label{alg:importRCode}
   \caption{Import Your R Modules.}
\SetAlgoLined
1. \code{source\_path =} \codeGreen{"Users/Documents/project1/kirtis\_class\_core.R"} \;
2. \code{source(source\_path)} \;
3. Use the function \codeCyan{diversify} from \code{R\_engine\_core.R} like you use any function of R: \\
\code{diversified\_portfolio = \codeCyan{diversify}(\codeGoldenrod{x}, \codeGoldenrod{y}, \codeGoldenrod{z})}
\end{algorithm}
\end{tcolorbox}


\begin{rem}
Please note there are two standard conventions for 
variable names where the variable name is consist of
more than one word. Either use an underscore or first
letter of each word (starting from the second word) is capitalized;
\codeGreen{evi\_ratio} or \codeGreen{eviRatio}.
\end{rem}








\chapter{Google Earth Engine}
\label{chap:Google Earth Engine}

\section{Preface}
Google Earth Engine sucks! 
Below (Fig.~\ref{fig:GEESucks}) 
we have a simple example to show GEE is very specific.
Accessing to elements/entries of its object is not
intuitive. Figuring out every single step is a 
challenge.
\begin{figure}
  \centering
  \includegraphics[width=1\textwidth]{figures/GEE_Sucks}
  \caption{GEE sucks.}
  \label{fig:GEESucks}
\end{figure}
\noindent Here is the code used for generation of Fig.~\ref{fig:GEESucks}.
\begin{tcolorbox}
%  \textbf{Algorithm for gradient descent with momentum is given below}
  \begin{algorithm}[H]
  \label{alg:GEE_Sucks}
   \caption{GEE Sucks.}
\SetAlgoLined
% \KwResult{Faster convergence}
1. \func{print}(\codeRed{``Print \#1: 3+10''},~\vari{3+10}) \;
2. \textbf{var} \code{x=}\vari{3}  \;
3. \textbf{var} \code{y=}\vari{10} \;
4. \func{print}(\codeRed{``Print \#2: x+y''}, 
\code{x+y}) \;
\vspace{.1in}
5. \textbf{var} \code{big\_delta\_x} = \vari{3} \;
6. \func{print}(\codeRed{``Print \#3: big\_delta\_x''}, \code{big\_delta\_x}) \;
\vspace{.1in}
7. \textbf{var} \code{x\_big} = \func{ee.List.sequence}(\vari{-125.0},~\vari{-111.3}, \code{big\_delta\_x}) \;
\vspace{.1in}
8. \func{print} (\codeRed{``Print \#4: x\_big''}, \code{x\_big}) \;
9. \func{print}(\codeRed{``Print \#5: x\_big.get(1)''}, \code{x\_big}.\func{get}(\vari{1})) \;
10. \func{print}(\codeRed{``Print \#6''},

\func{\textbf{ee.Number}}(\code{big\_delta\_x}).\func{add}
(\func{\textbf{ee.Number}}(\code{x\_big}.\func{get}(\vari{1})))) \;
\vspace{.1in}
11. \textbf{var} \code{aaa} 
     = \code{x\_big}.\func{get}(\vari{1}) +
     \code{big\_delta\_x} \;
12. // \func{print}(\codeRed{``Print \#7: aaa''}, \code{aaa}) \; 
13. \func{print}(\codeRed{``Print \#8: ee.Number(aaa)''}, \func{\textbf{ee.Number}}(\code{aaa})) \;
\end{algorithm}
\end{tcolorbox}

%%%%%%%----------------------------------------
%%%%%%%
%%%%%%%  JS or Python Interface
%%%%%%%
\section{JavaScript or Python Interface}
I think Python should be avoided in
this particular case for the following reasons:
\begin{enumerate}
    \item The interface is too slow,
    \item The interface needs authentication every 
          single time,
    \item Google does not maintain the Python. Therefore,
          the functions are first written/updated for
          the JavaScript (JS) by Google,
          and the Python
          equivalents/updates will not be provided 
          in a timely manner (who knows when?).
    \item The tutorials for JS is already 
          hard to find, it is much worse for Python. 
          Again, since Google is responsible for
          JavaScript, it releases the
          tutorials for it, but not Python.
          
          P.S. tutorials for JS might be abundant,
          but finding your exact needs might be hard.
          Even when you find something you may not
          be sure if that is the best possible
          solution.
\end{enumerate}

%%%%%%%--------------------------------------------------------------------------
%%%%%%%
%%%%%%%  Landsat Products and Differences
%%%%%%%
There are different products\footnote{start here to collect some information. some of the products are deprecated and superseded and Google does not show them easily: \href{https://developers.google.com/earth-engine/datasets/catalog/landsat}{here}} that fall under 
different labels; tier 1 vs tier 2, collection 1 vs collection 2, level 1 and level 2.
Some of these have the same description on 
Google developer pages. For example, 
\href{https://developers.google.com/earth-engine/datasets/catalog/LANDSAT_LC08_C01_T1_SR#description}{USGS Landsat 8 Surface Reflectance Tier 1}
and \href{https://developers.google.com/earth-engine/datasets/catalog/LANDSAT_LC08_C01_T2_SR#description}{USGS Landsat 8 Surface Reflectance Tier 2}
have the same description and identical bands. In this particular example
we want to use Tier 1. But we need a deeper understanding of differences(?)

Based on the information below and references therein, Collection 2 is an improvement
over Collection 1\footnote{Is there any time period for which Collection 2 
does not exist but 1 does?}.
It seems Collection-2 Level-2 Tier-1 should be the best, but
in our plots it was not different from T1\_SR (\cref{fig:C2L2Performance}). 
Also keep in mind \hl{Collection-2 Level-2 bands must be scaled.}

\begin{figure}[htb]
 \includegraphics[width=1\textwidth]{figures/00_merged_Landsats_Smoothed_and_raw}
\caption{In this plot the data points from Landsat-5, -7, and -8 (Tier 1, Surface Reflectance, from GEE collection LANDSAT/LE07/C01/T1\_SR) are merged together to form one vector. The same is done to Landsat-5 and -7 Collection-2 Level-2 (from GEE collection LANDSAT/LE07/C02/T1\_L2). We can see they all are performing well.} 
\label{fig:C2L2Performance}
\end{figure}

Moreover, GEE~\citep{Landsat7T1SRBandWidths}
says ``This dataset is the atmospherically corrected surface
reflectance from the Landsat 7 ETM+ sensor.'' about ``USGS Landsat 7 
Surface Reflectance Tier 1''  (LANDSAT/LE07/C01/T1\_SR). 
On the other hand, it also says ``Caution: 
This dataset has been superseded by LANDSAT/LC08/C02/T1\_L2.'' 

Collection-1 has only Level-1 data, however, Collection-2 has 
level-1 as well as Level-2.

\begin{description}
\item [Collection 1] Landsat Collection 1 was 
established in 2016 to improve archive management.
\href{https://www.usgs.gov/core-science-systems/nli/landsat/landsat-collection-1?qt-science_support_page_related_con=1#qt-science_support_page_related_con}{Learn more about Collection 1 from the USGS.}

Landsat Collection 1 consists of Level-1 data products 
generated from Landsat 8 Operational Land Imager (OLI)/Thermal 
Infrared Sensor (TIRS), Landsat 7 Enhanced Thematic Mapper 
Plus (ETM+), Landsat 4-5 Thematic Mapper (TM)*, and Landsat 1-5 
Multispectral Scanner (MSS) instruments.
\textbf{Collection 1 Tiers:}
    \begin{description}
    \item [Tier 1]
    ``Landsat scenes with the highest available data quality are placed into 
    Tier 1 and are considered suitable for time-series analysis.''~\citep{C1Describe}
    
    \item [Tier 2] ``Landsat scenes not meeting Tier 1 criteria during processing
     are assigned to Tier 2. Tier 2 scenes adhere to the same radiometric 
     standard as Tier 1 scenes, but do not meet the Tier 1 geometry 
     specification due to less accurate orbital information 
     (specific to older Landsat sensors), significant cloud cover, 
     insufficient ground control, or other factors.''~\citep{C1Describe}
    \end{description}
\item [Collection 2] 
Landsat Collection 2 marks the second major reprocessing effort on 
the Landsat archive by the USGS that results in several data product 
improvements that harness recent advancements in data processing, 
algorithm development, and data access and distribution capabilities. 
\href{https://www.usgs.gov/core-science-systems/nli/landsat/landsat-collection-2?qt-science_support_page_related_con=1#}{Learn more about Collection 2 from the USGS.}

Collection-2 Level-1 has different processings for different satellites~\citep{C2L1Describe}.
It seems Collection-2 level-1 is TOA and Collection-2 level-2 is Surface Reflectance. 
``Collection-2 Level-2 science products are generated from Collection 2 Level-1 
inputs that meet the <76 degrees Solar Zenith Angle constraint and include the 
required auxiliary data inputs to generate a scientifically viable 
product.''~\citep{C2L2Describe}.
``\textbf{Surface reflectance} (unitless) 
measures the fraction of incoming solar radiation that is 
reflected from the Earth's surface to the Landsat sensor. 
The LEDAPS and LaSRC surface reflectance algorithms 
correct for the temporally, spatially and spectrally varying scattering 
and absorbing effects of atmospheric gases, aerosols, and water vapor, 
which is necessary to reliably characterize the Earth’s land surface.''~\citep{C2L2Describe}.
For the enhancement details please see~\citep{C2L2Describe}.
\end{description}


\section{Scaling the Bands}
\label{sec:Scaling-the-Bands}
The purpose of this section is to make a point.
Since it is an important point, a section is devoted to it.

If you look at the band tables on \href{https://developers.google.com/earth-engine/datasets/catalog/COPERNICUS_S2#bands}{Sentinel-2},
there is a column called \emph{scale}. If you look at the band
table of \href{https://developers.google.com/earth-engine/datasets/catalog/LANDSAT_LC08_C02_T1_L2#bands}{Landsat 8 Level 2, Collection 2, Tier 1}, there are two columns called \emph{scale} and \emph{offset}. But such columns do not exist on 
\href{https://developers.google.com/earth-engine/datasets/catalog/LANDSAT_LT05_C01_T1_TOA#bands}{Landsat 5 TM Collection 1 Tier 1 TOA Reflectance}.

For some reason, Google Earth Engine has not scaled the bands
and has made that your problem. So, you have to scale the bands
properly during computations. If you forget to scale in case of Sentinel-2 and 
$\NDVI = \frac{\NIR - R}{\NIR + R}$ you will be lucky since scales
cancel out but that will not happen in case of EVI because 
of the additional 1 in the denominator 
(or in case of Landsat an off-set parameter is present as well);
\begin{gather}\label{eq:EVIeq} % gather and aligned leads to having one label for eq.
\begin{aligned}
\EVI &\coloneqq  G \times \frac{\rho_{NIR} - \rho_R}{\rho_{NIR} + C_1 \rho_R - C_2 \rho_B + L} \\ 
   &= 2.5 \times \frac{\rho_{NIR} - \rho_R}{\rho_{NIR} + 6 \rho_R - 7.5 \rho_B + 1} \\ 
\end{aligned}
\end{gather} 

Moreover, if you search the web for masking clouds in Sentinel,
you will find the function \href{https://developers.google.com/earth-engine/datasets/catalog/COPERNICUS_S2}{maskS2clouds}. 
If you look closely, in the last line the function is dividing 
the result by 10,000. Therefore, you do not have to scale
the bands again in computation of VIs. However,
you have to apply the \func{maskS2clouds} functions
to the image collection before computing the VIs.

%%%%%%%--------------------------------------------------------------------------
%%%%%%%
%%%%%%%  Access a Feature/Entry of a FeatureCollection
%%%%%%%
\section{Access a Feature/Entry of a FeatureCollection}

Suppose your \code{featurecollection} 
is called \code{SF}.
In order to access its entries you have to
convert it to a \vari{list} and then use \func{get(.)}:\\

\func{print} (\codeRed{``SF.get(0)''},
\code{SF}.\func{toList}(\vari{4}).\func{get}(\vari{0}));\\

\noindent where \vari{4} is the size 
of \code{SF} known in advance, and
\vari{0} is index of first entry of \code{SF}. 
In general you can use:\\

\func{print} (\codeRed{``SF.get(0)''},
\code{SF}.\func{toList}(\vari{\code{SF}.\func{size}\code{()}}).\func{get}(\vari{index}));\\

\noindent Please note if you use 
\code{SF}.\func{get}(\codeCyan{0}) 
you will get an error.

%%%%%%%----------------------------------------
%%%%%%%
%%%%%%% 
%%%%%%%
\section{Add a Property to a Feature}

Suppose you have uploaded a shapefile \vari{SF}
into your assets. The shapefiles usually have a 
component/slice called \vari{data} (which is of type datatable) 
that can be
accessed via \vari{SF@data} in R. This component stores
metadata corresponding to each polygon.

Say each polygon is an agricultural field
that has some attributes associated with it such as irrigation type,
area of the field, etc. After some computations
on GEE you may want to attach these metadata to
the output to use later. These metadata is referred to
by \vari{properties} on GEE. If you want to manually add a
property to a feature you should use:\\

\code{a\_feature = a\_feature}.\func{set}(\codeRed{`my\_property'}, \codeRed{1});\\

If you want to copy \vari{properties} (metadata)
of \vari{feature\_b} into \vari{feature\_a} you can do:\\

\code{feature\_a} = \code{feature\_a}.\func{copyProperties}(\code{feature\_b}, [\codeRed{`ID'}, \codeRed{`Irrigation\_type'}]);\\

\noindent where [\codeRed{`ID'}, \codeRed{`Irrigation\_type'}] 
is a subset of \vari{properties} of \vari{feature\_b} 
to be copied into \vari{feature\_a}.
I guess if that argument is dropped, then
all \vari{properties} will be copied.

%%%%%%%----------------------------------------
%%%%%%%
%%%%%%% Find Centroid of Polygons
%%%%%%%
\section{Find Centroid of Polygons}
Suppose you have a shapefile that you have
uploaded to GEE as an \emph{asset}. Here
we will see how to find the centroids of
the polygons in the shapefile.
Let the name of shapefile be \vari{Our\_ShapeFile}.
The function to compute centroids of the
polygons in \vari{Our\_ShapeFile} is
given by Alg.~\ref{alg:FindCentroidsAPoly}\footnote{This algorithm is 
accessible on GEE \href{https://code.earthengine.google.com/df1685205d4fbfd8def0efb12a75f8e4}{here}.}.
Line 4 of the Alg.~\ref{alg:FindCentroidsAPoly}
is keeping the columns of data slice in \vari{Our\_ShapeFile}; 
\vari{Our\_ShapeFile}@data.
\begin{tcolorbox}
  \begin{algorithm}[H]
  \label{alg:FindCentroidsAPoly}
   \caption{Find Centroids of Polygons in a Shapefile.}
\SetAlgoLined
% \KwResult{Faster convergence}
1. \textbf{function} \code{getCentroid(feature)} \{ \\
  \vspace{.1in}
\hspace{.2in} 2. {\color{ForestGreen}{// Keep this list of properties.}}\;
\hspace{.2in} 3. \textbf{var} \code{keepProperties =} [\codeRed{`ID'}, \codeRed{`county'}] \;
  \vspace{.2in}
\hspace{.2in} 4. {\color{ForestGreen}{// Get the centroid of the feature's geometry.}}\;
\hspace{.2in} 5. \textbf{var} \code{centroid =} 
\code{feature}.\func{geometry}().\func{centroid}(); \;
  \vspace{.2in}
\hspace{.2in} 6. 
{\color{ForestGreen}{// Return a new Feature, 
copying properties from the
  
\hspace{0.5in} old Feature.}}\;
\hspace{.2in} 7. \textbf{return}
\func{ee.Feature}(\code{centroid}).\func{copyProperties}
     (\code{feature},
 
  \hspace{2.8in}\code{keepProperties})\;
8. \}
  
  \vspace{.1in}
9. \textbf{var} \code{SF =} \func{ee.FeatureCollection}(\code{Our\_ShapeFile})\;
10. \textbf{var} \code{centroids\_from\_GEE} = \code{SF}.\func{map}(\code{getCentroid});
\end{algorithm}
\end{tcolorbox}

{\color{red}{Warning:}} 
Imagine your polygon looks like a doughnut (non-convex shape).
Then the centroid would be in the center 
of the disk in the center
of the doughnut which is not part of the doughnut/polygon/region of interest.
So, if you want to look at an area around 
the centroid, then
that area (or parts of it, depending on how large the area is) 
would not belong to the polygon 
(See Fig.~\ref{fig:badBuffer}; it is not a doughnut, 
but it delivers the message!)
\begin{figure}
\subfloat[][Bad Centroid]{\includegraphics[width=0.5\textwidth]{figures/badCentroid}\label{fig:badCentroid}} \qquad
\subfloat[][Bad Buffer]{\includegraphics[width=0.5\textwidth]{figures/badBuffer}\label{fig:badBuffer}}
\caption[][5\baselineskip]{Centroids and buffers around the centroids of 
         polygons in a shapefile.
\label{fig:badPolygon}}
\end{figure}

By adding one line (line 5.5 in Alg.~\ref{alg:FindCentroidsBuffers}) 
to the function \func{getCentroid(.)}
we can get a buffer (a rectangular or a circle area) around
the centroids.
\begin{tcolorbox}
  \begin{algorithm}[H]
  \label{alg:FindCentroidsBuffers}
   \caption{Make a Buffer Around Centroids of Polygons.}
\SetAlgoLined
% \KwResult{Faster convergence}
1. \textbf{function} 
\code{get\_rectangle\_arround\_centroid(feature)}\{ \\
  \vspace{.1in}
\hspace{.2in} 2.  {\color{ForestGreen}{// Keep this list of properties.}}\;
\hspace{.2in} 3. \textbf{var} \code{keepProperties} = [\codeRed{`ID'}, \codeRed{`county'}] \;
  \vspace{.2in}
\hspace{.2in} 4. {\color{ForestGreen}{// Get the centroid of the feature's geometry.}}\;
\hspace{.2in} 5. \textbf{var} 
\code{centroid} = 
\code{feature}.\func{geometry}().\func{centroid}(); \;
\vspace{.1in}
  
\hspace{.2in} 5.5 
\small{\code{centroid} =
\func{ee.Feature}(\code{centroid}.\func{buffer}(\codeCyan{200}).\func{bounds}())}\;

\vspace{.1in}
\hspace{.2in} 6. {\color{ForestGreen}{// Return a new Feature, copying properties from the
  
\hspace{.2in} old Feature.}}\;
\hspace{.2in} 7. \textbf{return} \func{ee.Feature}(\code{centroid}).\func{copyProperties}(\code{feature},
  
  \hspace{2.8in} 
  \code{keepProperties})\;
8. \}

  \vspace{.1in}
9. \textbf{var} \code{SF} =
\func{ee.FeatureCollection}(\code{Our\_ShapeFile})\;
10. \textbf{var}
\code{centroids\_from\_GEE} =  

\hspace{1.4in}
\code{SF}.\func{map}
(\code{get\_rectangle\_arround\_centroid});
\end{algorithm}
\end{tcolorbox}

%%%%%%%----------------------------------------
%%%%%%%
%%%%%%%       Cloud Filtering
%%%%%%%
\section{Cloud Filtering}
Handling clouds for Sentinel and Landsat are different. 
Let us start by \textbf{Sentinel}.\\

\noindent First, the followings are equivalent:

\begin{itemize}[leftmargin=0.5cm]
 \item  \textbf{var} \code{filtered = my\_IC}.\func{filterMetadata}(
    \codeRed{`CLOUDY\_PIXEL\_PERCENTAGE'},

    \hspace{2.5in} \codeRed{`less\_than'}, \codeRed{70});
    
    \item \textbf{var} \code{filtered = my\_IC}.\func{filter}(\codeRed{`CLOUDY\_PIXEL\_PERCENTAGE < 70'})
    
    \item \textbf{var} \code{filtered = my\_IC}.\func{filter}(\func{ee.Filter.lte}(\codeRed{`CLOUDY\_PIXEL\_PERCENTAGE'}, 
    
    \hspace{2.9in} \codeRed{70}))
\end{itemize}
\noindent They all filter out \emph{images} 
with cloud cover less than or equal to 70\%.
Those images will NOT be in our 
\code{filtered} collection.
Said differently, 
our \code{filtered} 
collection may include images that 
are covered by cloud up t0 70\%.

This is a pre-filtering step. Later, 
we can toss out the cloudy 
\emph{pixels} from every single image.
\begin{tcolorbox}
  \begin{algorithm}[H]
  \label{alg:FilterCloudyPixelsSentinel}
   \caption{Filter Cloudy Pixels for Sentinel.}
\SetAlgoLined
% \KwResult{Faster convergence}
1. \textbf{function} \code{maskS2clouds(image)} \{ \\
  \vspace{.2in}
\hspace{.2in} 2. {\color{ForestGreen}{// Each Sentinel-2 
                 image has a bitmask band with cloud 

\hspace{.55in} mask information QA60.}}\;
\hspace{.2in} 
3. \textbf{var} \code{qa = image}.\func{select}(\codeRed{`QA60'}); 

\vspace{.2in}
\hspace{.2in} 
4. {\color{ForestGreen}{// Bits 10 and 11 
                are clouds and cirrus, respectively.}}\;

\hspace{.2in} 5. \textbf{var} \code{cloudBitMask =} \codeCyan{1} $\code{<<}$ \codeCyan{10}  \;

\hspace{.2in} 6. \textbf{var} \code{cirrusBitMask =} \codeCyan{1} $\code{<<}$ \codeCyan{11} \;

\vspace{.2in}
\hspace{.2in} 7. 
{\color{ForestGreen}{// Both flags 
should be set to zero, 
indicating clear 

\hspace{.6in} conditions.}}\;

\hspace{.2in} 8. 
\textbf{var} \code{mask = qa}.\func{bitwiseAnd}(\code{cloudBitMask})
                  .\func{eq}(\codeCyan{0}).\func{and}(

\hspace{1.1in}     \code{qa}.\func{bitwiseAnd}(\code{cirrusBitMask})
                   .\func{eq}(\codeCyan{0}))\;

\vspace{.2in}
\hspace{.2in} 9. {\color{ForestGreen}{// Return the masked 
                 and scaled data, without 
        
                  \hspace{.7in} the QA bands.}}

\hspace{.2in} 10. 
\textbf{return}
\code{image}.\func{updateMask}(\code{mask})

\hspace{1.25in}
.\func{divide}(\codeCyan{10000})
                
\hspace{1.25in}
.\func{select}(\codeRed{``B.*''})
                
\hspace{.4in} .\func{copyProperties}(
                \code{image}, [\codeRed{``system:time\_start''}]);
                
\hspace{.01in} 11. \}
\end{algorithm}
\end{tcolorbox}

\noindent \textbf{{\color{red}{Note 1:}}} Please note
the last line in Alg.~\ref{alg:FilterCloudyPixelsSentinel}
is copying the system start time into the image which
has nothing to do with clouds. It may be handy later.\\


\noindent \textbf{{\color{red}{Note 2:}}} Please note
the three (equivalent) pre-filtering of images mentioned 
above do not exist for Landsat!\\

Landsat(s) is a different satellite, and 
therefore, the cloud filtering must 
be handled
differently; the band names that includes 
cloud information are different 
between Sentinel and
Landsat or even among different Landsats.\\

Landsat-8 \emph{Surface Reflectance} cloud mask~\citep{Landsat8CloudMask}:
\begin{tcolorbox}
  \begin{algorithm}[H]
  \label{alg:FilterCloudyPixelsLandsat8}
   \caption{Filter Cloudy Pixels for Landsat-8 \textbf{Tier 1 and 2} \emph{Surface Reflectance}.}
\SetAlgoLined
% \KwResult{Faster convergence}
1. \textbf{function} 
\code{maskL8sr(image)} \{\\

\vspace{.1in}

\hspace{.2in} 2. 
{\color{ForestGreen}{// Bits 3 and 5 are cloud 
shadow and cloud, 

\hspace{.55in} respectively.}}\;

\hspace{.2in} 3. \textbf{var}
\code{cloudShadowBitMask} =
(\codeCyan{1} $\code{<<}$ \codeCyan{3})\;
\hspace{.2in} 4. \textbf{var} \code{cloudsBitMask} = (\codeCyan{1} $\code{<<}$ \codeCyan{5})\;

\vspace{.2in}  
\hspace{.2in} 5. {\color{ForestGreen}{// Get the pixel QA band.}}\;

\hspace{.2in} 6.  
\textbf{var} \code{qa =} \code{image}.\func{select}(\codeRed{`pixel\_qa'})\;

\vspace{.2in}
\hspace{.2in} 7. {\color{ForestGreen}{// Both flags should be set to zero, indicating clear 

\hspace{.5in}  conditions.}}\;

\hspace{0.2in} 8. {\small{
\textbf{var} \code{mask = }
          \code{qa}.\func{bitwiseAnd}(\code{cloudShadowBitMask})
          .\func{eq}(\codeCyan{0})
          
\hspace{1.25in} .\func{and}(\code{qa}.\func{bitwiseAnd}(\code{cloudsBitMask}).\func{eq}(\codeCyan{0}))}}\;

\vspace{.2in}
\hspace{.2in} 9. \textbf{return}
\code{image}.\func{updateMask}(\code{mask})\;
  
\hspace{.01in} 10. \}
\end{algorithm}
\end{tcolorbox}


\noindent \textbf{{\color{red}{Note:}}} This is
written for Landsat-8 (Surface Reflectance Tier 1 and 2).\\

The code for masking the cloudy pixels in Landsat-4, 5, and 7 
\emph{Surface Reflectance} is given by~\citep{Landsat457CloudMask}
that is given below by Alg.~\ref{alg:FilterCloudyPixelsLandsat457}:
\begin{tcolorbox}
  \begin{algorithm}[H]
  \label{alg:FilterCloudyPixelsLandsat457}
   \caption{Filter Cloudy Pixels for Landsat-4, 5, and 7 \textbf{Tier 1 and 2} \emph{Surface Reflectance}.}
\SetAlgoLined
% \KwResult{Faster convergence}

1. \textbf{function} 
\code{cloudMaskL457(image)} \{

\vspace{.1in}
\hspace{.2in} 2. 
\textbf{var} \code{qa = image}.\func{select}(\codeRed{`pixel\_qa'})\;

\vspace{.1in}
\hspace{.2in} 3. {\color{ForestGreen}{// If the cloud bit (5) is set and
                      the cloud confidence (7) 
                      
\hspace{0.6in}   is high or the cloud shadow bit is set (3), 


\hspace{0.6in}        then it's a bad pixel.}}\\

\hspace{.2in} 4. 
\textbf{var} 
\code{cloud = qa}.\func{bitwiseAnd}(\codeCyan{1} $\code{<<}$ \codeCyan{5})\\
\hspace{1.35in}   .\func{and}(\code{qa}.\func{bitwiseAnd}(\codeCyan{1} $\code{<<}$ \codeCyan{7}))\\
\hspace{1.35in}   .\func{or}(\code{qa}.\func{bitwiseAnd}(\codeCyan{1} $\code{<<}$ \codeCyan{3}))\;

\vspace{.1in} 

\hspace{.2in} 5.
{\color{ForestGreen}{// Remove edge pixels that don't occur in all bands}}\\

\hspace{.2in} 6.
\textbf{var} 
\code{mask2 = image}.\func{mask}().\func{reduce}(\func{ee.Reducer.min}())\;

\hspace{.2in} 7.
\small{\textbf{return} \code{image}.\func{updateMask}(\code{cloud}.\func{not}()).\func{updateMask}(\code{mask2});}
  
\hspace{.01in} 10. \}
\end{algorithm}
\end{tcolorbox}

I have copied the cloud masking functions
from GEE development/data-product pages into a script that can be found 
\href{https://code.earthengine.google.com/?scriptPath=users\%2Fhnoorazar\%2FGEE_Mini_Tutorial\%3ACloudMaskings}{here}~\citep{CloudandMaskingFunctionsonMyGEE}.
More on masking clouds of Sentinel-2 and shadows
are provided \href{https://developers.google.com/earth-engine/tutorials/community/sentinel-2-s2cloudless}{here}~\citep{CloudandShadowSentinel} by GEE developers.


Another way of masking cloud used in~\citep{rs11070820}:

\begin{tcolorbox}
  \begin{algorithm}[H]
  \label{alg:FilterCloudyPixelsLandsat78TOA}
   \caption{Filter Cloudy Pixels for Landsat-7 and 8 \emph{TOA};
   LANDSAT/LC08/C01/T1\_TOA and\\
   LANDSAT/LE07/C01/T1\_TOA.}
\SetAlgoLined
% \KwResult{Faster convergence}

1. \textbf{function} 
\code{cloudMask(image)} \{

\vspace{.1in}
\hspace{.2in} 2. 
\textbf{var} \code{cloudscore =} ee.Algorithms.Landsat\\\hspace{1.6in} \func{.simpleCloudScore}(image).\\ 
\hspace{1.6in}  \func{.select}(`cloud')\;

\vspace{.1in}

\hspace{.2in} 3.
\small{\textbf{return} \code{image}.\func{updateMask}(cloudscore.\func{lt}(50));}
  
\hspace{.01in} 10. \}
\end{algorithm}
\end{tcolorbox}


%%%%%%%----------------------------------------
%%%%%%%
%%%%%%%     Timelines
%%%%%%%
\section{Timelines}
\label{sec:Timelines}
Figure~\ref{fig:landsatTimeline}
shows the timeline of Landsat
satellites~\citep{LandsatTimelinesWiki}
and~\cref{tab:ElevationTable} shows the
exact dates.
\begin{figure}[htb]
  \includegraphics[width=1\textwidth]{figures/landsatTimeline1}
\caption{Landsat Timeline.}
\label{fig:landsatTimeline}
\end{figure}

\begin{table}[]
\centering
\caption{Landsat timeline table.} 
\label{tab:ElevationTable}
\begin{tabular}{|l|l|l|}
\hline
\rowcolor{shadecolor} 
\small{Satellite} & 
\small{Launched} & 
\small{Terminated} \\
\hline
Landsat 5 & 1 March 1984 & 5 June 2013 \\ 
\hline
\rowcolor{aliceblue} 
Landsat 6 & 5 October 1993  & 5 October 1993 \\ \hline
Landsat 7 & 15 April 1999 & Still active \\ \hline
\rowcolor{aliceblue} 
Landsat 8 & 11 February 2013 & Still active \\ \hline
Landsat 9 & 16 September 2021 (planned) & - \\ \hline
\end{tabular}
\end{table}

%%%%%%%----------------------------------------
%%%%%%%
%%%%%%%     Band Names and Indices
%%%%%%%
\section{Band Names and Indices}
\label{sec:Band_Names_and_Indices}
Band names are different in each instrument
(see~\cref{tab:BandNameTable}).
Hence the indices must be defined differently
using proper band names.
Below we see some of indices.
\cref{tab:SomeBandWavelengths} also provides
more insight about the bandwidths of the satellites.
The bandwidths are very similar. If their
minimal differences makes any difference I am not aware of it and do not care. 
Go nuts if you wish; figure out why, what, how.
Bandwidths of Sentinel-2 is found on Wikipedia~\cite{SentinelBandwidths}
and Bandwidths of Landsats can be found on GEE pages (e.g.~\citep{Landsat7T1SRBandWidths}).

\begin{table}[]
\centering
\caption{Some Band Names in Satellites.} 
\label{tab:BandNameTable}
\begin{tabular}{|l|l|l|l|}
\hline
\rowcolor{shadecolor} 
\small{Satellite} & 
\small{NIR} & 
\small{Red}  &
\small{Blue} \\
\hline
Sentinel & B8 & B4 & B2\\ 
\hline
\rowcolor{aliceblue} 
Landsat-8 & B5 & B4 & B2\\ 
\hline
Landsat-7 & B4 & B3 & B1\\ 
\hline
\rowcolor{aliceblue} 
Landsat-5 & B4 & B3 & B1\\ 
\hline
\end{tabular}
\end{table}


\begin{table}[]
\centering
\caption[][9\baselineskip]{Some Band Wavelengths. The bandwidths are very similar. If their
minimal differences makes any difference I am not aware of it and do not care. 
Go nuts if you wish; figure out why, what, how.} 
\label{tab:SomeBandWavelengths}
\begin{tabular}{|l|l|l|l|}
\hline
\rowcolor{shadecolor} 
\small{Satellite} & 
\small{NIR} & 
\small{Red}  &
\small{Blue} \\
\hline
Sentinel-2A & B8:~~~~~~0.77 -- 0.88 $\mu m$ & B4:~~~~~~0.65 -- 0.68 $\mu m$ & B2:~~~~~~0.46 -- 0.52 $\mu m$\\ 
\hline
\rowcolor{aliceblue} 
Sentinel-2B & B8:~~~~~~0.78 -- 0.88 $\mu m$ & B4:~~~~~~0.65 -- 0.68 $\mu m$ & B2:~~~~~~0.46 -- 0.52 $\mu m$ \\ 
\hline
Landsat-8 & B5:~~~~~~0.85 -- 0.88 $\mu m$ & B4:~~~~~~0.64 -- 0.67 $\mu m$ & B2:~~~~~~0.45 -- 0.51 $\mu m$\\ 
\hline
\rowcolor{aliceblue} 
Landsat-7 & B4:~~~~~~0.77 -- 0.90 $\mu m$ & B3:~~~~~~0.63 -- 0.69 $\mu m$ & B1:~~~~~~0.45 -- 0.52 $\mu m$\\ 
\hline
Landsat-5 & B4:~~~~~~0.77 -- 0.90 $\mu m$  & B3:~~~~~~0.63 -- 0.69 $\mu m$ & B1:~~~~~~0.45 -- 0.52 $\mu m$\\ 
\hline
\rowcolor{aliceblue} 
Landsat-7 C2 L2 & SR\_B4: 0.77 -- 0.90 $\mu m$ & SR\_B3: 0.63 -- 0.69 $\mu m$ & SR\_B1: 0.45 -- 0.52 $\mu m$\\ 
\hline
Landsat-5 C2 L2 & SR\_B4: 0.77 -- 0.90 $\mu m$ & SR\_B3: 0.63 -- 0.69 $\mu m$ & SR\_B1: 0.45 -- 0.52 $\mu m$\\ 
\hline
\end{tabular}
\end{table}


\begin{gather}
\label{eq:EVILandsat8}
\begin{aligned}
\EVI &= G \times \frac{\NIR - R}{\NIR + C1 \times R - C2 \times B + L} \\
\EVI_S &= 2.5 \times \frac{B8 - B4}{B8 + 6 \times B4 - 7.5 \times B2 + 1}\\
\EVI_8 &= 2.5 \times \frac{B5 - B4}{B5 + 6 \times B4 - 7.5 \times B2 + 1} \\
\EVI_7 &= 2.5 \times \frac{B4 - B3}{B4 + 6 \times B3 - 7.5 \times B1 + 1}
\end{aligned}
\end{gather} 

\noindent where $NIR$ is near infrared, $R$ is Red,
$B$ is blue, 
$\text{EVI}_8$ is the Enhanced Vegetation Index (EVI) 
in Landsat-8~\citep{Landsat8EVI}, 
and $\text{EVI}_S$
is the EVI in Sentinel; The NIR band in
Landsat-8 is $B5$~\citep{L8BandNames}
and for Sentinel is $B8$.

``EVI is similar to Normalized Difference 
Vegetation Index (NDVI) and can be used 
to quantify vegetation greenness. However, 
EVI corrects for some atmospheric conditions 
and canopy background noise and is more 
sensitive in areas with dense vegetation. 
It incorporates an ``$L$'' value to adjust for 
canopy background, ``$C$'' values as coefficients 
for atmospheric resistance, and values from 
the blue band ($B$).  These enhancements allow 
for index calculation as a ratio between 
the $R$ and $NIR$ values, while reducing the 
background noise, atmospheric noise, and 
saturation in most cases''~\citep{Landsat8EVI}.

Below are the NDVIs for 
Landsat-4 to Landsat-7~\citep{Landsat4NDVI},
Landsat-8~\citep{Landsat4NDVI}, and Sentinel:
\begin{gather}
\label{eq:NDVILandsat8}
\begin{aligned}
\NDVI &= \frac{\NIR - R}{\NIR + R}\\
\NDVI_S &= \frac{B5 - B4}{B5 + B4}\\
\NDVI_8 &= \frac{B8 - B4}{B8 + B4} \\
\NDVI_{4-7} &= \frac{B4 - B3}{B4 + B3} \\
\end{aligned}
\end{gather}

Landsat-7 has 8-day NDVI composite already provided 
by GEE~\citep{Landsat7NDVIComposite}. This product
is based on TOA data which is not perfect! However,
it seems running some smoothing methods on it can
make it useful.


\section{Tiny Tips, Big Problems}
\label{sec:Tiny-Tips-Big-Problems}
The tips in this section are useful for beginners and
if you want to do something that is unusual.

Some times you may find yourself in a situation
for which you are using the biggest sledgehammer to deal
with the tiniest nail. In these scenarios the empire of Google
does not have a function (for good reasons most likely) to do the job. 
If brute force is the chosen approach then these tips may be handy. 
If you are the only person on the planet who wants
to do a certain thing, maybe you need to think again,
and let go of useless approaches.

\begin{description}
\item [Object Types] There are two types of objects or functions.
Some are called server-side. Some are called client-side.
Here is an \href{https://code.earthengine.google.com/?scriptPath=users\%2Fhnoorazar\%2FGEE_Mini_Tutorial\%3AobjectTypeForLoop}{example}  
that shows a client-side object does not work with server-side object.

It is strongly advised to avoid using/writing client-side
objects/functions. The client-side objects also make the
server/code/interface be very slow, freeze at times.

\item [Batch Export] This is an example that Google does not think
is useful. But if you need to export a collection of images you can do it
either using a for-loop for which you may need to look at the previous example.
Or, you can use \func{batch.Download.ImageCollection.toDrive(.)}. 
Both of these approaches are demonstrated \href{https://code.earthengine.google.com/?scriptPath=users\%2Fhnoorazar\%2FGEE_Mini_Tutorial\%3ABatchExport}{here}.

Two remarks in this regard. First, the function for downloading the image 
collection as a batch\footnote{\func{batch.Download.ImageCollection.toDrive(.)}} 
behaves strangely.\footnote{I was visualizing the images as RGB 
images and exporting them; \textbf{var} 
\code{imageRGB =} \codeCyan{an\_image}. 
\func{visualize}(vizParams). I am not too sure if the 
batch download's problem is specific to RGB images.}
In~\cref{fig:strangeBatch} there are 4 parts. The top left shows
two images in a folder; one is exported via for-loop and the other
is exported via batch-download. In the batch-downloaded image,
naked eye cannot see anything, it is black and white. After opening it,
it turns all into white (lower left). But the image exported via for-loop
can be seen with naked eye (top right). The strange event is that the 
batch-downloaded image, can be seen if it is opened via Python or 
GIS (lower right image)!

\begin{figure}[htb]
  \centering
  \includegraphics[width=.9\textwidth]{figures/strangeBatch}
  \caption{Strange Behavior of Batch-Download.}
  \label{fig:strangeBatch}
\end{figure}

The images I exported turned out to be black and white.
Secondly, any time a data is exported on GEE interface, you need
to click on the \textbf{Run} button on \textbf{Task} tab. Perhaps Python
can be used to avoid this problem, as well as server-side/client-side
problem altogether.

\item [Enough is a Feast] Only choose the bands you need.
The reason is that when you apply an operation to the image collections,
it is applied to all bands.
For example, you may want to take maximum value of NDVI per pixel in an image collection. When you use \func{qualityMosaic(.)} function, it does not know
your intention. It is applied to all bands. This can be expensive and
take time. Moreover, suppose you merge data from Landsat 5, 7, and 8
on Google Earth Engine to use \func{qualityMosaic(.)} on the merged collection.
Some additional bands are present in Landsat 8 that are not present in
Landsat 5. Therefore, \func{qualityMosaic(.)} will not know what to do
with those bands.
Therefore, after 3 days of waiting you will get an error!
\end{description}

\section{Full Code Examples; Landsat and Sentinel}

Here are two examples, one for Landsat-8~\citep{FullCodeLandsat8} 
and one for Sentinel-2~\citep{FullCodeSentinel}.
They are both on the GEE Mini Tutorial repo on 
Google Earth Engine~\citep{GEEMiniTutorialRepoonEE}. 
I have had problems with sharing repo in the past.
If that does not work, you can copy the codes
from the GitHub repo where this PDF is located
at~\citep{MiniTutorialOnGitHub}.
There is also a shapefile on Google drive~\citep{ShapeFileOnDrive} that is used
in some of these codes.

\subsection{Code Example}
Merging data on GEE; see appendix of the paper\\
~\href{https://www.mdpi.com/2072-4292/11/7/820/htm}{https://www.mdpi.com/2072-4292/11/7/820/htm}

\section{Some More Remarks}
\label{sec:SomeMoreRemarks}

\begin{description}
\item [Filter Clouds and Scale.]
Please look at the way clouds and shadows are
filtered and proper bands are scaled with not hard-coding
on the GEE page\cite{MaskCloudShadowScaleprepSrL8}
in the function \func{prepSrL8(.)}.

\item [Merge on GEE] You can merge
two image collections (e.g. image collections from Landsat 7 and 8)
on GEE like so \\
\code{var merge\_IC = ee.ImageCollection(col\_1.}\func{merge}\code{(col\_2))};

\end{description}

\section{Beyond GEE}
I like to advise once you are done with GEE,
read your CSV files (Python/R/etc.) and round the 
digits to 2 (or 3?) decimal places
if you will and re-write them on the disk. That reduces the file sizes.
Of course this matters when you are working
with substantial number of fields.




\bibliography{GEE_References}

\end{document}