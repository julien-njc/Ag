\documentclass[12pt]{article}
\usepackage{amsmath}
\usepackage{amsthm}
\usepackage{esvect}
\usepackage[toc,page]{appendix}
\usepackage{leftidx}
\usepackage{color}
\usepackage{framed, color}
\usepackage{multirow}
\usepackage{pdfpages} % makes trouble in JASSS
\usepackage{multicol}
\usepackage{wrapfig,lipsum,booktabs}

\usepackage[utf8]{inputenc}
\usepackage{mathtools,hyperref} % mathtools make trouble in JASSS
\usepackage{cleveref}
\usepackage{commath}

\usepackage{enumitem}
\usepackage{amssymb}
\renewcommand{\qedsymbol}{$\blacksquare$}
%%%%%%%%%%%
\usepackage{mathtools,hyperref}
\hypersetup{
    colorlinks=true,
    linkcolor=cyan,
    filecolor=cyan,      
    urlcolor=red,
    citecolor=red,
}
%%%%%%%%%%%%%%%%%% Pseudo Code package
\usepackage[linesnumbered,ruled]{algorithm2e}

\usepackage[x11names, dvipsnames]{xcolor}
\usepackage{lipsum}

\newlength{\seplinewidth}
\newlength{\seplinesep}
\setlength{\seplinewidth}{1mm}
\setlength{\seplinesep}{2mm}
\colorlet{sepline}{PaleVioletRed3}
\newcommand*{\sepline}{%
  \par
  \vspace{\dimexpr\seplinesep+.5\parskip}%
  \cleaders\vbox{%
    \begingroup % because of color
      \color{sepline}%
      \hrule width\linewidth height\seplinewidth
    \endgroup
  }\vskip\seplinewidth
  \vspace{\dimexpr\seplinesep-.5\parskip}%
}
%%%%%%%%%%%%%%%%%%%%%%%%%%%
%%%%%%%%%%%%%%%%%%%%%%%%%%%
%%%%%%%%%%%%%%%%%%%%%%%%%%%
\def\code#1{\textbf{\texttt{#1}}}
\def\vari#1{{\color{Cerulean}{\textbf{\texttt{#1}}}}}
\def\func#1{{\color{Blue}{\textbf{\texttt{#1}}}}}
\newenvironment{coded}{\color{blue}\code}


%%%%%%%%%%%
%%%%%%%%%%%  User Defined Commands. (macros)
%%%%%%%%%%%
\newcommand{\T}{\mathbb{T}}
\newcommand{\Op}{\mathbb{O}}
\definecolor{mgreen}{RGB}{25,147,100}
\definecolor{shadecolor}{rgb}{1,.8,.1}
\definecolor{shadecolor2}{RGB}{245,237,0}
\definecolor{orange}{RGB}{255,137,20}
\definecolor{orange}{RGB}{245,37,100}
\newcommand\ddfrac[2]{\frac{\displaystyle #1}{\displaystyle #2}}


\usepackage[capposition=bottom]{floatrow}
\usepackage{capt-of}
\usepackage{booktabs}
\usepackage{varwidth}
\usepackage[T1]{fontenc}
\usepackage[font=small,labelfont=bf,tableposition=top]{caption}
\DeclareCaptionLabelFormat{andtable}{#1~#2  \&  \tablename~\thetable}

%%%%%%%%%%%
%%%%%%%%%%%  Graphical Packages
%%%%%%%%%%%
\usepackage{mdframed}
\usepackage{adjustbox}
\usepackage{tcolorbox}
\usepackage{graphics}
\usepackage{tikz,ifthen,fp,calc} % calc makes trouble

\usepackage{subcaption}
\captionsetup[figure]{font=small,labelfont={bf,sf}}
\captionsetup[subfigure]{font=scriptsize,labelfont={bf,sf}}

\usetikzlibrary{plotmarks}
\usepackage{graphicx}
\usepackage{capt-of}% or \usepackage{caption}
\usepackage{booktabs}
\usepackage{varwidth}


%%%%%%%%%%%%%%%%%
%%%%%%%%%%%%%%%%% Theorem Styles
%%%%%%%%%%%%%%%%%
\theoremstyle{plain}
\newtheorem{theorem}{Theorem}[section]
\newtheorem{prop}{Proposition}[section]
\newtheorem{corr}{Corollary}[section]
\theoremstyle{definition}
\newtheorem{definition}{Definition}[section]
\newtheorem{lemma}[theorem]{Lemma}
\theoremstyle{definition}
\newtheorem{remark}{Remark}[section]
\newtheorem{fact}{Fact}[section]

\usepackage[english]{babel}
\usepackage{babel,blindtext}
\newtheorem{corollary}{Corollary}[theorem]
\newtheorem{exmp}{Example}[section]
\usepackage{fullpage}
\usepackage{amsfonts}
\usepackage{lscape}
\usepackage{bbm}

\usepackage{todonotes}
\usepackage{cite}
\usepackage{verbatim}
\usepackage{bm}

%%%%%%%%%%%%%%%%%%%%%%%%%%%% alphabeticall order of citations
\usepackage[numbers]{natbib}
%%%%%%%%%%%%%%%%%%%%%%%%%%%% double space
%%%%%%%%%%%%%%%%%%%%%%%%%%%%
\newcommand*{\affaddr}[1]{#1} % No op here. Customize it for different styles.
\newcommand*{\affmark}[1][*]{\textsuperscript{#1}}

\usepackage{authblk}
\DeclareMathOperator*{\argmax}{arg\,max}
\usepackage[margin=.5in]{geometry}
\title{Codling Moth Pseudo Code}

\author[1]{Hossein Noorazar \thanks{h.noorazar@wsu.edu}}

\providecommand{\keywords}[1]{\textbf{\textit{Keywords:---}} #1}
%\pgfplotsset{compat=1.12}

\usepackage[utf8]{inputenc}
\date{}
\begin{document} 
\maketitle
\date{}

\vspace{-.6in}

\begin{algorithm}[httb!]
    \SetKwInOut{Input}{Input}
    \SetKwInOut{Output}{Output}

    \Input{\normalsize{\vari{raw\_file, params, start\_year, end\_year, lower=10, upper=31.11}}}

    \Output{\normalsize{\vari{CMPOP\_files (minus some columns such as ClimateGroup, latitude, etc.)}}}
    \vspace{.1in}
    
     \code{ \normalsize prepare\_time\_stuff(start\_year, end\_year)};       \hspace{.1in} //{\normalsize{Nyears, Nrecords, NofVariables, Years, ind}}\\
    
    \code{ \normalsize create\_ymdvalues (nYears, Years, leap.year)}  \hspace{.1in}  // {\normalsize{Generate Calendar}}\\
    
     \code{ \normalsize readbinarydata\_addmdy(input\_file, Nrecords, Nofvariables, ymd, ind)}
    
     \code{ \normalsize add\_dd\_cumdd(data, lower, upper)}  \hspace{.1in}  // {\normalsize{Calculate daily and cumulative gdd}}\\
    
     \code{ \normalsize add day of year from 1 to 365/366}\\
    
     \code{ \normalsize CodlingMothRelPopulation(params, data)} \hspace{.1in}  // {\normalsize{compute relative population}}\\
    
     \code{ \normalsize append relative population to the columns of the data: (tmax, tmin, dd, cum\_dd, cum\_dd\_F)}
    
     \code{\normalsize rename some of the columns: "SumEgg", "SumLarva", "SumPupa", "SumAdult", "dayofyear", 
                      "year", "month", "day"}
                      
     \code{\normalsize CodlingMothPercentPopulation(params, data)}  \hspace{.1in}  // {\normalsize{Compute percentage population}}\\
    
    \normalsize \code{\normalsize append percentage pop. to the rest of the data}\\
    \code{\normalsize return data}    
    
    \caption{Generate CMPOP files}
    \label{alg:prepareDataCMPOP}
\end{algorithm}
%%%%%%%%%%%%%%%%%%%%%%%%%%%%%%%%%%%%%%%%%%%%%%%%%%%%%%%%%%%%%%%%%%%%%%%%%%%%%%%%%%%%%%%%%%%
%%%%%%%%%%%%%%%%%%%%%%%%%%%%%%%%%%%%%%%%%%%%%%%%%%%%%%%%%%%%%%%%%%%%%%%%%%%%%%%%%%%%%%%%%%%
%%%%%%%%%%%%%%%%%%%%%%%%%%%%%%%%%%%%%%%%%%%%%%%%%%%%%%%%%%%%%%%%%%%%%%%%%%%%%%%%%%%%%%%%%%%
%%%%%%%%%%%%%%%%%%%%%%%%%%%%%%%%%%%%%%%%%%%%%%%%%%%%%%%%%%%%%%%%%%%%%%%%%%%%%%%%%%%%%%%%%%%
\begin{algorithm}[httb!]
    \SetKwInOut{Input}{Input}
    \SetKwInOut{Output}{Output}
    \Input{\normalsize{\vari{input\_data, param, start\_year, end\_year, lower=10, upper=31.11}}}
    \Output{\normalsize{\vari{CM files}}}
    \vspace{.1in}
    \code{ \normalsize prepare\_time\_stuff(start\_year, end\_year)};       \hspace{.1in} //{\normalsize{Nyears, Nrecords, NofVariables, Years, ind}}\\
    
    \code{ \normalsize create\_ymdvalues (nYears, Years, leap.year)}  \hspace{.1in}  // {\normalsize{Generate Calendar}}\\
    
     \code{ \normalsize readbinarydata\_addmdy(input\_file, Nrecords, Nofvariables, ymd, ind)}
    
     \code{ \normalsize add\_dd\_cumdd(data, lower, upper)}  \hspace{.1in}  // {\normalsize{Calculate daily and cumulative gdd}}\\
    
     \code{ \normalsize add day of year from 1 to 365/366}\\
     
      \code{ \normalsize add cumulative DD in celsius to data}\\
      
      \code{ \normalsize compute relative population and append it to the data. (CodlingMothRelPopulation(params, metdata))}\\
      
       \code{ \normalsize compute percentage population and append it to the data. (CodlingMothPercPopulation(params, metdata))}\\
       
       \code{ \normalsize compute generations of adults and larva of all 4 generations by the beginning of each month  and append it to the data.}\\
       
       \code{ \normalsize compute emergence and diapause  and append it to the data.}\\
      
      \code{ \normalsize compute when the 25\%, 50\% and 75\% of generations are hit.}\\
     
     \code {return CM\_file}
    \caption{Generate CM files (prepareDataCMPOP)}
    \label{alg:prepareDataCM}
\end{algorithm}

%%%%%%%%%%%%%%%%%%%%%%%%%%%%%%%%%%%%%%%%%%%%%%%%%%%%%%%%%%%%%%%%%%%%%%%%%%
%%%%%%%%%%%%%%%%%%%%%%%%%%%%%%%%%%%%%%%%%%%%%%%%%%%%%%%%%%%%%%%%%%%%%%%%%%
%%%%%%%%%%%%%%%%%%%%%%%%%%%%%%%%%%%%%%%%%%%%%%%%%%%%%%%%%%%%%%%%%%%%%%%%%%
%%%%%%%%%%%%%%%%%%%%%%%%%%%%%%%%%%%%%%%%%%%%%%%%%%%%%%%%%%%%%%%%%%%%%%%%%%
\begin{algorithm}[httb!]
    \SetKwInOut{Input}{Input}
    \SetKwInOut{Output}{Output}
    
    \Input{\normalsize{\vari{params, metdata}}}
    \Output{\normalsize{\vari{percentage\_population}}}
    \vspace{.1in}
    
     \code{ \normalsize read parameters and generate an empty data frame with columns (dayofyear, year, month, Cum\_dd\_F)} \\
    
       \normalsize \For{$k = 1, \ldots, stage\_gen\_toiterate$ }{%
       \normalsize relnum $\gets$ pweibull(data[Cum\_dd\_F], shape=params[i, 3], scale= params[i,4]) \\
       \normalsize{add proper column name such as perc\_egg\_gen\_1}\\
      }
      
      generate columns (PercEgg, PercLarva, PercPupa, PercAdult) \\
      
      \normalsize \For{ all columns of data frame }{%
       \normalsize allrelnum\$PercEgg[allrelnum[Cum\_dd\_F] $>$ params [i,5] \& 
                      allrelnum[Cum\_dd\_F] $<=$ params [i,6]] $\gets$ allrelnum[allrelnum[Cum\_dd\_F] > params [i, 5] \& 
                                                                              allrelnum[Cum\_dd\_F] $<=$ params [i,6], 
                                                                              columnnumber]
      }
      
       return \textbf{allrelnum}
    \caption{Codling Moth Percentage Population (CodlingMothPercPopulation)}
    \label{alg:CodlingMothPercentPopulation}
\end{algorithm}



In the above algorithm stage\_gen\_toiterate is 16, 4 generations of eggs, 4 generations of larva, 4 of pupaes, 4 of adults.
%%%%%%%%%%%%%%%%%%%%%%%%%%%%%%%%%%%%%%%%%%%%%%%%%%%%%%%%%%%%%%%%%%%%%%%%%%
%%%%%%%%%%%%%%%%%%%%%%%%%%%%%%%%%%%%%%%%%%%%%%%%%%%%%%%%%%%%%%%%%%%%%%%%%%
%%%%%%%%%%%%%%%%%%%%%%%%%%%%%%%%%%%%%%%%%%%%%%%%%%%%%%%%%%%%%%%%%%%%%%%%%%
%%%%%%%%%%%%%%%%%%%%%%%%%%%%%%%%%%%%%%%%%%%%%%%%%%%%%%%%%%%%%%%%%%%%%%%%%%

\begin{algorithm}[httb!]
    \SetKwInOut{Input}{Input}
    \SetKwInOut{Output}{Output}
%    \underline{function adjacencyGeneration} $(n_s,n_p,u_b)$\;
    \Input{\normalsize{\vari{params, metdata}}}
    \Output{\normalsize{\vari{relative\_population}}}
    \vspace{.1in}
    \normalsize choose a subset of data (day\_of\_year, month, year, cumdd\_F)\\
     \For{ all of the stages }{%
     relnum $\gets$ dweibull(metdata[cumdd\_F], shape= params[i, 3], scale= params[i, 4]) * 10000\\
     attach it to the data with proper name.
      }
      
      
      \For {all stages such as eggs}{
      data [SumEgg] = data[EggGen1] + data[EggGen2] + data[EggGen3] + data[EggGen4]
       }
       return \textbf{data}
    \caption{Codling Moth Relative Population (CodlingMothRelPopulation)}
    \label{alg:CodlingMothRelPopulation}
\end{algorithm}
%%%%%%%%%%%%%%%%%%%%%%%%%%%%%%%%%%%%%%%%%%%%%%%%%%%%%%%%%%%%%%%%%%%%%%%%%%%%%%%%%%%%%%%%%%%
%%%%%%%%%%%%%%%%%%%%%%%%%%%%%%%%%%%%%%%%%%%%%%%%%%%%%%%%%%%%%%%%%%%%%%%%%%%%%%%%%%%%%%%%%%%
%%%%%%%%%%%%%%%%%%%%%%%%%%%%%%%%%%%%%%%%%%%%%%%%%%%%%%%%%%%%%%%%%%%%%%%%%%%%%%%%%%%%%%%%%%%
%%%%%%%%%%%%%%%%%%%%%%%%%%%%%%%%%%%%%%%%%%%%%%%%%%%%%%%%%%%%%%%%%%%%%%%%%%%%%%%%%%%%%%%%%%%
\begin{algorithm}[httb!]
    \SetKwInOut{Input}{Input}
    \SetKwInOut{Output}{Output}
    \Input{\normalsize{\vari{metdata, lower, upper}}}
    \Output{\normalsize{\vari{metdata with additional columns}}}
    \vspace{.1in}
    Compute the degree days and cumulative degree days according those 6 type of relations between tmin, tmax, lower and upper temps.\\
    
     return \textbf{data}
     
    \caption{Add cumulative Degree Days (add\_dd\_cumdd)}
    \label{alg:add_dd_cumdd}
\end{algorithm}
%%%%%%%%%%%%%%%%%%%%%%%%%%%%%%%%%%%%%%%%%%%%%%%%%%%%%%%%%%%%%%%%%%%%%%%%%%%%%%%%%%%%%%%%%%%
%%%%%%%%%%%%%%%%%%%%%%%%%%%%%%%%%%%%%%%%%%%%%%%%%%%%%%%%%%%%%%%%%%%%%%%%%%%%%%%%%%%%%%%%%%%
%%%%%%%%%%%%%%%%%%%%%%%%%%%%%%%%%%%%%%%%%%%%%%%%%%%%%%%%%%%%%%%%%%%%%%%%%%%%%%%%%%%%%%%%%%%
%%%%%%%%%%%%%%%%%%%%%%%%%%%%%%%%%%%%%%%%%%%%%%%%%%%%%%%%%%%%%%%%%%%%%%%%%%%%%%%%%%%%%%%%%%%
\begin{algorithm}[httb!]
    \SetKwInOut{Input}{Input}
    \SetKwInOut{Output}{Output}
    \Input{\normalsize{\vari{combined\_CMPOP, lower\_temp = 4.5, upper\_temp = 24.28}}}
    \Output{\normalsize{\vari{metdata with additional columns}}}
    \vspace{.1in}
    generate vertical degree days\\
    group by long, lat, climate scenario, climate group, year to generate cumulative vert. DD\\
    generate 3 new columns for three type of apples (\code{pnorm(data[vert\_cumdd\_F], mean = 495.51, sd = 42.58, lower.tail = TRUE)})
     return \textbf{data}
     
    \caption{(generate\_vertdd)}
    \label{alg:generatevertdd}
\end{algorithm}
\vspace{-10in}
%%%%%%%%%%%%%%%%%%%%%%%%%%%%%%%%%%%%%%%%%%%%%%%%%%%%%%%%%%%%%%%%%%%%%%%%%%%%%%%%%%%%%%%%%%%
%%%%%%%%%%%%%%%%%%%%%%%%%%%%%%%%%%%%%%%%%%%%%%%%%%%%%%%%%%%%%%%%%%%%%%%%%%%%%%%%%%%%%%%%%%%
%%%%%%%%%%%%%%%%%%%%%%%%%%%%%%%%%%%%%%%%%%%%%%%%%%%%%%%%%%%%%%%%%%%%%%%%%%%%%%%%%%%%%%%%%%%
%%%%%%%%%%%%%%%%%%%%%%%%%%%%%%%%%%%%%%%%%%%%%%%%%%%%%%%%%%%%%%%%%%%%%%%%%%%%%%%%%%%%%%%%%%%
\begin{algorithm}[httb!]
    \SetKwInOut{Input}{Input}
    \SetKwInOut{Output}{Output}
    \Input{\vari{combined\_CMPOP.RDS}}
    \Output{\vari{Absolute and relative population of diapause}}
    \vspace{.1in}

   Look at the code please.
     
    \caption{Diapause, absolute and relative population (diapause\_abs\_rel)}
    \label{alg:diapauseabsrel}
\end{algorithm}
%%%%%%%%%%%%%%%%%%%%%%%%%%%%%%%%%%%%%%%%%%%%%%%%%%%%%%%%%%%%%%%%%%%%%%%%%%%%%%%%%%%%%%%%%%%
%%%%%%%%%%%%%%%%%%%%%%%%%%%%%%%%%%%%%%%%%%%%%%%%%%%%%%%%%%%%%%%%%%%%%%%%%%%%%%%%%%%%%%%%%%%
%%%%%%%%%%%%%%%%%%%%%%%%%%%%%%%%%%%%%%%%%%%%%%%%%%%%%%%%%%%%%%%%%%%%%%%%%%%%%%%%%%%%%%%%%%%
%%%%%%%%%%%%%%%%%%%%%%%%%%%%%%%%%%%%%%%%%%%%%%%%%%%%%%%%%%%%%%%%%%%%%%%%%%%%%%%%%%%%%%%%%%%
\begin{algorithm}[httb!]
    \SetKwInOut{Input}{Input}
    \SetKwInOut{Output}{Output}
    \Input{\vari{combined\_CMPOP.RDS}}
    \Output{\vari{Absolute and relative population of diapause}}
    \vspace{.1in}

   Look at the code please.
     
    \caption{Diapause, absolute and relative population (diapause\_abs\_rel)}
    \label{alg:diapauseabsrel}
\end{algorithm}


\vspace{-10in}
%%%%%%%%%%%%%%%%%%%%%%%%%%%%%%%%%%%%%%%%%%%%%%%%%%%%%%%%%%%%%%%%%%%%%%%%%%%%%%%%%%%%%%%%%%%
%%%%%%%%%%%%%%%%%%%%%%%%%%%%%%%%%%%%%%%%%%%%%%%%%%%%%%%%%%%%%%%%%%%%%%%%%%%%%%%%%%%%%%%%%%%
%%%%%%%%%%%%%%%%%%%%%%%%%%%%%%%%%%%%%%%%%%%%%%%%%%%%%%%%%%%%%%%%%%%%%%%%%%%%%%%%%%%%%%%%%%%
%%%%%%%%%%%%%%%%%%%%%%%%%%%%%%%%%%%%%%%%%%%%%%%%%%%%%%%%%%%%%%%%%%%%%%%%%%%%%%%%%%%%%%%%%%%
\begin{algorithm}[httb!]
    \SetKwInOut{Input}{Input}
    \SetKwInOut{Output}{Output}
    \Input{\vari{vertdd\_combined\_CMPOP\_.RDS}}
    \Output{\normalsize{\vari{bloom}}}
    \vspace{.1in}
    
    Look at the code please
     
    \caption{Bloom (bloom)}
    \label{alg:bloom}
\end{algorithm}



%%%%%%%%%%%%%%%%%%%%%%%%%%%%%%%%%%%%%%%%%%%%%%%%%%%%%%%%%%%%%%%%%%%%%%%%%%%%%%%%%%%%%%%%%%%
%%%%%%%%%%%%%%%%%%%%%%%%%%%%%%%%%%%%%%%%%%%%%%%%%%%%%%%%%%%%%%%%%%%%%%%%%%%%%%%%%%%%%%%%%%%
%%%%%%%%%%%%%%%%%%%%%%%%%%%%%%%%%%%%%%%%%%%%%%%%%%%%%%%%%%%%%%%%%%%%%%%%%%%%%%%%%%%%%%%%%%%
%%%%%%%%%%%%%%%%%%%%%%%%%%%%%%%%%%%%%%%%%%%%%%%%%%%%%%%%%%%%%%%%%%%%%%%%%%%%%%%%%%%%%%%%%%%
\iffalse
\begin{algorithm}[httb!]
\tiny
    \SetKwInOut{Input}{Input}
    \SetKwInOut{Output}{Output}
%    \underline{function adjacencyGeneration} $(n_s,n_p,u_b)$\;
    \Input{Two nonnegative integers $n_s$, $n_p$ and $u_b \in [0,1]$}
    \Output{adjacency matrix \textbf{A}}
    $N = n_s \times n_p$;                     \hspace{.1in} //population size\\
    \textbf{A} = zero matrix of size $N$    \hspace{.1in}  // initiate adjacency matrix\\
    // Initiate diagonal blocks:\\
    \For{$k = 1, \ldots, n_s$ }{%
       $a^k_{ij} = 1/(n_p-1)$ \hspace{.1in} if $ i \neq j $
      }
       // generate off-diagonal blocks:\\ 
       \For{$i = 2, \ldots, n_s$ }{%
          \For{$j = 1, \ldots, i-1$ }
          {%
          $\textbf{A}_{ij}$  choose randomly from $[0,u_b)$ \\
          $\textbf{A}_{ji} \gets \textbf{A}_{ij}$ // copy lower diagonal blocks to upper diagonal blocks\\
           }
      }
      %% start While
      \While{not converged}
      {
      %% start For
      \For{$rowCount = 1, \ldots, \text{N}$}
      {%
      // divide each row by sum of its entries:\\
      $\textbf{A}[rowCount,:] \gets \textbf{A}[rowCount,:]/sum(\textbf{A}[rowCount,:])$
      }%
     %% End For

     %% start For
      \For{$colCount = 1, \ldots, \text{N}$}
      {%
      // divide each column by sum of its entries:\\
      $\textbf{A}[:, \: colCount ] \gets \textbf{A}[:,colCount]/sum(\textbf{A}[:,colCount])$\
      }%
     %% End For
            \For{$i = 2, \ldots, n_s$ }{%
          \For{$j = 1, \ldots, i$ }
          {%
          $\textbf{A}_{ji} \gets \textbf{A}_{ij}$ // copy lower diagonal blocks to upper diagonal blocks\\
           }
      }
      } %% End While
      return \textbf{A}
    \caption{\small Random adjacency matrix generation.(RAMG)}
    \label{alg:adjacencyGenAlg}
\end{algorithm}

\fi





\end{document}







